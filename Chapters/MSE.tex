%************************************************
\chapter{Maximal Static Expansion}\label{ch:MSE}
%************************************************

\section{Motivations}
Data-dependencies in a program can lead to a very bad automatic parallelization. Modern compilers use techniques to reduce the number of such dependencies. One of them is \ac{MSE}. The \ac{MSE} is a transformation which expands the memory accesses to and from Array or Scalar. The goal is to disambiguate memory accesses by assigning different memory locations to non-conflicting writes. This method is described in a paper written by Denis Barthou, Albert Cohen and Jean- Francois Collard\cite{MSE}. Let take an example (from the article) to understand the principle :
\begin{lstlisting}[frame=single]
int tmp;
for (int i = 0; i < N; i++) {
    tmp = i;
    for (int j = 0; j < N; j++) {
        tmp = tmp + i + j;
    }
    A[i] = tmp;
}
\end{lstlisting}

The data-dependences induced by tmp make the two loops unparallelizable : the iteration j of the inner-loop needs value from the previous iteration and it is impossible to parrallelize the i-loop because tmp is used in all iterations. If we expand the accesses to tmp according to the outermost loop, we can then parallelize the i-loop.

\begin{lstlisting}[frame=single]
int tmp_exp[N];
for (int i = 0; i < N; i++) {
    tmp_exp[i] = i;
    for (int j = 0; j < N; j++) {
        tmp_exp[i] = tmp_exp[i] + i + j;
    }
    A[i] = tmp_exp[i];
}
\end{lstlisting}

The accesses to tmp are now made to/from a different location for each iteration of the i-loop. It is then possible to execute the different iterations on different computation units (\ac{GPU}, \ac{CPU} …).

We will present in the next sections two differents methods to expand memory accesses : Fully-Indexed and Maximal Static expansion.

\section{Fully-Indexed Expansion}
The principle of \ac{FIE} is that each write goes to a different memory location. This expansion is the maximum one in term of memory needed. Depending on the characteristics of the code expanded, it could be the fast expansion or the worst but never the best one in terms of memory consumption.

The mechanism to fully expand the memory accesses is to expand the write accesses according to the loop induction variables and map the read accesses to the newly expanded arrays. Let see that on an example :

\begin{lstlisting}[frame=single]
int tmp;
for (int i = 0; i < N; i++) {
    tmp = i;
    for (int j = 0; j < N; j++) {
S:      B[j] = tmp + 3;
    }
T:  A[i] = B[i];
}
\end{lstlisting}

There are two write accessses :
\[
\{T[i] \rightarrow A[i]:0<=i<=N\}
\]
This one is already fully expanded. So no need of expansion for the array A.

\[
\{S[i,j] \rightarrow B[j]:0<=i<=N,0<=j<=N\}
\]
The statement S is inside a two dimensional loop nest and access the array B that is a one-dimension array. So the array B needs expansion. We then create a new 2D array called $B_{exp}$, indexed with the loop induction variables $i$ and $j$.

The write access of B has been modified so it is necessary to remap the read access of B to $B_{exp}$. To get the memory location read by a statement, dependency analysis is needed. The idea to remap accesses is to analyse \ac{RAW} dependencies between statements. For example, in our code, there is a \ac{RAW} dependency between T and S :
\[
\{[T[i] \rightarrow B[i]] \rightarrow [S[i,i] \rightarrow B[i,i]]:0<=i<=N\}
\]
This \ac{ISL} map means that the read access made during the execution of statement T[i] reads a memory location written by statement S[i][i]. The new read access must be :
 \[
\{T[i] \rightarrow B[i][i]:0<=i<=N\}
\]

Here is the final expanded example :
\begin{lstlisting}[frame=single]
int tmp;
for (int i = 0; i < N; i++) {
    tmp = i;
S:  for (int j = 0; j < N; j++) {
        B_exp[i][j] = tmp + 3;
    }
T:  A[i] = B_exp[i][i];
}
\end{lstlisting}


\section{Maximal Static Expansion}
\ac{FIE} is efficient to remove dependencies but inefficient in term of memory consumption. Moreover, \ac{FIE} can require a $\phi$-function to disambiguate some accesses. A $\phi$-function is a function that say, according to run-time information, where to read/write. Let take a small example to illustrate :

\begin{lstlisting}[frame=single]
for (int i = 0; i < N; i++) {
S:  tmp = 1
    if( i%2 ) {
T:     tmp = 0;
    }
U:  ... = tmp
}
\end{lstlisting}

Depending on the iteration we are in, the statement U can read the value of tmp from the location of tmp written by S or T. In our case, the $\phi$-function is :

\[ \phi(U) = \begin{cases} 
      S & \text{if $i\%2$} \\
      T & \text{if not $i\%2$}
   \end{cases}
\]

The use of $\phi$-function creates a need of test at run time to decide where to read/write and thus create an overhead in term of execution time. The goal of maximal expansion is to have a tradeof between parallelism and memory consumption while not having the need of a $\phi$-function. 

We will not discuss about how to implement \ac{MSE} as soon as, due to time constraint, I was not able to implement it. But details can be found in the article\cite{MSE}. But let take an example to illustrate the difference between \ac{FIE} and \ac{MSE}. Parts with point have no impact on the value of tmp.

\begin{lstlisting}[frame=single]
for (int i = 0; i < N; i++) {
S:  tmp = ...
    while( c ) {
T:      tmp = tmp + ...
    }
U:  ... = tmp;
}
\end{lstlisting}

The statement U reads the value of tmp but the source is only known at run time, depending on the condition of the while loop. If we fully expand this code, we will have :

\begin{lstlisting}[frame=single]
for (int i = 0; i < N; i++) {
    tmp1[i] = ...
    int j = 0;
    while( c ) {
        if (j == 0)
          tmp2[i][j] = tmp1[i] + ...
        else
          tmp2[i][j] = tmp2[i][j-1]
        j++
    }
    if (j ==0)
      ... = tmp1[i]
    else
      ... = tmp3[i][j]
}
\end{lstlisting}
We can see that there are two run-time conditions that will slow down the execution and maybe avoid parallelization. If we found the maximum expansion that can be done without needing a $\phi$-function, we have this code :

\begin{lstlisting}[frame=single]
for (int i = 0; i < N; i++) {
S:  tmp[i] = ...
    while( c ) {
T:      tmp[i] = tmp[i] + ...
    }
U:  ... = tmp[i];
}
\end{lstlisting}
This expansion is obviously not the best one in term of unicity of write location because the inner loop access the same element. But, we cannot do better without having a $\phi$-function. It is  possible to run in parallel each iteration of the outermost loop.