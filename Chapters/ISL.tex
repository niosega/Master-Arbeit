%************************************************
\chapter{Integer Set Library}\label{ch:ISL}
%************************************************

\ac{ISL}\cite{ISL}, is a C library handling sets and relations of integer points bounded by affine constraints. Sets and maps can be described with numerical values as well as with parameter values. The computations are done in exact integer arithmetic. 

There are multiple data structures inside ISL. Here is a quick overview of concepts needed to understand the next parts of this report. 

\section{Set}
A set in ISL is represented as follow :
\[
\{S[i_0,...,i_n]:c_0,...,c_n\}
\]
Where $i_k$ is an input variable and $c_k$ is a constraint.

For example, the set of all instances of the statement S of the following source code is $\{S[i,j]:i=j,0<=i<=N,0<=j<=N\}$.
\begin{lstlisting}[frame=single]
for (int i = 0; i < N; i++)
  for (int j = 0; j < N; j++)
    if (i == j)
S:    A[i][j+1] = i*j;
\end{lstlisting}

\section{Map}
A map in ISL represents a relation between two sets. It is represented as follow :
\[
\{S[i_0,...,i_n] \rightarrow T[o_0,...,o_n]:c_0,...,c_n\}
\]

Where $i_k$ is an input variable (also called input dimension), $o_k$ is an output variable (also called output dimension) and $c_k$ is a constraint.

For example, to model the memory access inside the statement S of the preeceding example, the corresponding map is :
\[
\{S[i,j] \rightarrow A[i,j+1]:i=j,0<=i<=N,0<=j<=N\}
\]
This map means that in the statement S, there is a memory access to the array A at the indices [i, j+1].

\textbf{BE CAREFUL :} there are implicit constraints in this map. An unsimplified version of this map is :

\[
\{S[i_0,i_1] \rightarrow A[o_0,o_1]:i_0=i_1,i_0=o_0,i_1=o_1-1\}
\]

The input part of the map is called the domain whereas the output part is called the range. 

For example, the domain and the range of the previous map are :
\[
domain(\{S[i,j] \rightarrow A[i,j+1]:i=j\})=\{S[i,j]:i=j\} 
\]
\[
range(\{S[i,j] \rightarrow A[i,j+1]:i=j\})=\{A[i,j+1]:i=j\}
\]

\section{Union Map and Union Set}
On a coarse-grain, an union map and an union set are just an union of map and of set.

\section{Nested Map}
There is a specific type map called nested map. The structure is following : 
\[
\{[S[i_0,...,i_n] \rightarrow T[j_0,...,j_m]] \rightarrow [U[k_0,...,k_r] \rightarrow V[l_0,...,l_t]]:c_0,...,c_a\}
\]

With this kind of data structure, we can represent data dependencies. Let take an example.

\begin{lstlisting}[frame=single]
for (int i = 0; i < N; i++) {
   for (int j = 0; j < N; j++) {
S:   B[i][j] = i*j;;
   }
T: A[i] = B[0][i];
}
\end{lstlisting}

There is a RAW dependency for the array B because we read index i of B in statement T after that the statement S has written to index j of B. The dependences map looks like :
\[
\{[T[i] \rightarrow B[0,i]] \rightarrow [S[i,j] \rightarrow B[i,i]]:0<=i<=N,0<=j<=N\}
\]

