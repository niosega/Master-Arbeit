%************************************************
\chapter{Implementation of Copy In/Out}\label{ch:CopyInOut}
%************************************************

Copy-in/out mechanism is not yet implemented, but we had few discussions about it. 

At first, we thought at a solution using memcpy intrisics present in LLVM-IR. But after a discussion with the Polly community, it turns out that memcpy is probably not a reliable solution : there are no guarantee that the existing memory layout directly corresponds to the expanded layout. And if it matches, LLVM has a LoopIdiom pass that will convert a loop that implements a memory copy to memcpy anyway.

To have a clear idea on how to implement Copy-in / out, let consider multiple cases :

\section{Copy-in of an llvm::Value}
This means the llvm::Value is defined outside of the SCoP. It has just one value during the SCoP, no expansion needed.

\section{Copy-in of PHI}
Let take an example :
\begin{lstlisting}[frame=single]
enter:
  %initval = ...
  br %scop_start

scop_start:
  %i = phi i32 [0, %enter], [%i.next, %scop_latch]
  %phi = phi double [%initval, %enter], [%x, %scop_latch]
  ...

scop_latch:
  %i.next = add %i, 1
  %x = ...
  br %scop_start,
\end{lstlisting}

\%phi can be expanded using \%i as index ("\%phi[\%i]"). If it is not coming from \%scop\_latch, it will have the value \%initval. That is, we can initialize all values of \%phi[\%i] with \%initval. It will be overwritten by \%x if not comming from \%scop\_latch. If we want to be precise, only \%phi[0] needs to be initiliazed to \%initval because the others memory cells will be overwritten anyway.

\section{Copy-in of an array...}
\subsection{... that is not written to inside the SCoP}
There are no write accesses, so no expansion is needed.

\subsection{... that is written to inside the SCoP}
By construction, we identify uses of an array element by the statement that wrote it. For instance:
\begin{lstlisting}[frame=single]
    for (int i = 0; i < n: i+=1)
S:  A[i] = ...;

    for (int i = 0; i < 2*n: i+=1)
T:   ... = A[i];
\end{lstlisting}

The dependencies for A would be:
\[
\{ Stmt\_for\_T[i] \rightarrow Stmt\_for\_S[i] : i \le n; Stmt\_for\_T[i] \rightarrow Copy\_in\_A[i] : i \ge n \}
\]

Copy\_ in\_ A is currently not part of what DependencyInfo returns. If implemented, we need to access it in T(i):
\begin{lstlisting}[frame=single]
    for (int i = 0; i < n: i+=1)
S:  Expanded_A_in_S[i] = ...;

    for (int i = 0; i < 2*n: i+=1)
T:   ... = (i < n) ? Expanded_A_in_S[i] : A[i];
\end{lstlisting}

That is, either the original A[i] for the copy-in, or the expanded one for S.

It can be considered a special case of :
\begin{lstlisting}[frame=single]
    for (int i = 0; i < 2*n: i+=1)
U:  A[i] = ...; /* init all array elements */

    for (int i = 0; i < n: i+=1)
S:  A[i] = ...;

    for (int i = 0; i < 2*n: i+=1)
T:   ... = A[i];
\end{lstlisting}

And the dependencies :
\[
\{ Stmt\_for\_T[i] \rightarrow Stmt\_for\_S[i] : i \le n; Stmt\_for\_T[i] \rightarrow Stmt\_for\_U[i] : i \ge n \}
\]

for which we need support for setNewAccessRelation(isl::union\_map). \\

It is not possible without because Expanded\_A\_in\_S[i] is expanded using S's domain. It doesn't have an element for $i \ge n$.

\section{Copy-out of PHINode}
Same as before, let take an example :
\begin{lstlisting}[frame=single]
scop_exiting:
  br i1 %exit_cond, label %scop_exit, label %stay_in_scop

scop_exit:
  %phi = double [%val, %scop_exiting] /* this is outside the SCoP */
\end{lstlisting}

\%scop\_exiting will have a WRITE MemoryAccess of MemoryKind::ExitPHI. It is written multiple times, but only the last is actually used. Using $isl\_union\_flow\_get\_{must/may}\_no\_source$ it is possible to find out which one is the last one, and only write that one.

\%phi is outside the scop, therefore there is no domain that could be used to expand it.

\section{Copy-out of arrays}

Again $isl\_union\_flow\_get\_{must/may}\_no\_source$ can be used to determine which statement instance has written the last value before leaving the SCoP. That element must be written to the array.

\begin{lstlisting}[frame=single]
for (int j = 0; j < m; i+=1)
  for (int i = 0; i < n; i+=1)
    A[i+j] = S(j,i);
\end{lstlisting}

What's visible in A[k] after the SCoP execution? After the execution of the SCOP, A[k] contains (k>=m) ? S(m-1,k-m-1): S(k,0).

Now we have two solutions to create copy-out statements.
\begin{itemize}
\item Create statements at the end of the SCOP.
\begin{lstlisting}[frame=single]
for (k = 0; k < n+m+1; k+=1)
  A[k] = Expanded_A_for_S[(k>=m) ? m-1 : k,(k>=m) ? k-m-1 : 0];
\end{lstlisting}

\item When doing the expansion, insert partial write that write to the original array, if needed.
\begin{lstlisting}[frame=single]
/* normal expansion without copy-out */
for (int j = 0; j < m; i+=1)
  for (int i = 0; i < n; i+=1)
    Expanded_A_for_S[i,k] = S(j,i);

/* with copy-out */
for (int j = 0; j < m; i+=1)
  for (int i = 0; i < n; i+=1) {
    double val = S(j,i); 
    Expanded_A_for_S[i,k] = val;
    if (i == 0 || j == m -1) 
       A[(j == m-1) ? i + m-1 : j] = val; 
 }
\end{lstlisting}
\end{itemize}



